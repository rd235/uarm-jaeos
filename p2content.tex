\setcounter {section} {1}
\section{Phase 2 - Level 3: The Nucleus}
Level 3 (the nucleus) of JaeOS builds on previous levels in two key ways:
\begin{itemize}
\item Building on the exception handling facility of Level 1 (the ROM-Excpt
handler), provide the exception handlers
that the ROM-Excpt handler “passes” exception handling “up” to. There will be
one exception handler for each type of exception: Program Traps (PgmTrap),
SYSCALL/\linebreak Breakpoint (SYS/Bp), TLB Management (TLB), and Interrupts (Ints).
\item Using the data structures from Level 2 (Phase 1), and the facility to handle 
both SYS/Bp exceptions and Interrupts - timer interrupts in particular - provide a process scheduler.
\end{itemize}
The purpose of the nucleus is to provide an environment in which asynchronous
sequential processes (i.e. heavyweight threads) exist, each making forward progress
as they take turns sharing the processor. Furthermore, the nucleus provides these
processes with exception handling routines, low-level synchronization primitives,
and a facility for “passing up” the handling of PgmTrap, TLB exceptions and certain SYS/Bp exceptions to the next level; the VM-I/O support level Level (Phase 3).
Since virtual memory is not supported by JaeOS until Level 4, all addresses at
this level are assumed to be physical addresses. Nevertheless, the nucleus needs to
preserve the state of each process. e.g. If a process is executing with virtual memory on 
(\verb+CP15_Control[0]=1+) when it is either interrupted or executes a
\verb+SYSCALL+,
	then \verb+CP15_Control+ should still be set to 1 when it continues its execution.
\subsection{The Scheduler}
Your nucleus should guarantee finite progress; consequently, every ready process
will eventually have an opportunity to execute. Processes in JaeOS belong to four
different priority levels: \verb+PRIO_LOW+, \verb+PRIO_NORM+,
\verb+PRIO_HIGH+ and \verb+PRIO_IDLE+. The latter is reserved for a special Idle Process that 
``twiddles its thumbs'' by endlessly putting the processor in a Wait State, waiting for an 
interrupt to occur. $\mu$ARM supports a \verb+WAIT+ ROM service expressly for this purpose. 

 The scheduler uses a 5 millisecond time slice selecting
the running process among the highest priority ready processes.
The scheduler also needs to perform some simple deadlock detection and if
deadlock is detected perform some appropriate action; e.g. invoke the
\verb+PANIC+
ROM service/instruction.
We define the following
\begin{itemize}
\item Ready Queues: Four queues of ProcBlk's (\verb+ struct list_head+)
representing processes that are ready and waiting for a turn at execution
(one queue per priority level).
\item Current Process: A pointer to a ProcBlk that represents the current executing process.
\item Process Count: The count of the number of processes in the system.
\item Soft-block Count: The number of processes in the system currently blocked
and waiting for an interrupt; either an I/O to complete, or a timer to “expire.”
\end{itemize}
The scheduler should behave in the following manner if the three main Ready Queues are
empty (i.e. the Idle Process has to be executed):
\begin{enumerate}
\item If the Process Count is zero invoke the \verb+HALT+ ROM service/instruction.
\item Deadlock for JaeOS is defined as when the Process Count $>$ 0 and the Soft-block 
Count is zero. Take an appropriate deadlock detected action. (e.g.
Invoke the \verb+PANIC+ ROM service/instruction.)
\item Continue execution loading the Idle Process.
\end{enumerate}
\subsection{Nucleus Initialization}
Every program needs an entry point (i.e. \verb+main()+), even JaeOS. The entry point
for JaeOS performs the nucleus initialization, which includes:
\begin{enumerate}
\item
Populate the four New Areas in the ROM Reserved Frame.
For each New processor state:
\begin{itemize}
\item Set the \verb+PC+ to the address of your nucleus function that is to handle
exceptions of that type.
\item Set the \verb+SP+ to RAMTOP. Each exception handler will use the last
frame of RAM for its stack.
\item Set the \verb+CPSR+ register to mask all interrupts and be in kernel-mode (System mode).
\end{itemize}
\item Initialize the Level 2 (Phase 1 - see Chapter 2) data structures:
\begin{verbatim}
							initPcbs()
							initASL()
\end{verbatim}
\item Initialize all nucleus maintained variables: Process Count, Soft-block Count,
Ready Queues, and Current Process.
\item Initialize all nucleus maintained semaphores. In addition to the above nucleus variables, there is one semaphore variable for each external (sub)device
in $\mu$ARM, plus a semaphore to represent a pseudo-clock timer. Since terminal devices are 
actually two independent sub-devices (see Section 5.7-pops), the nucleus maintains two semaphores for each terminal device. All
of these semaphores need to be initialized to zero.
\item Instantiate a single process and place its ProcBlk in the \verb+PRIO_NORM+ Ready Queue. A
process is instantiated by allocating a ProcBlk (i.e.
\verb+allocPcb()+), and
initializing the processor state that is part of the ProcBlk. In particular this
process needs to have interrupts enabled, virtual memory off, the processor
Local Timer enabled, kernel-mode on, \verb+SP+ set to RAMTOP-FRAMESIZE
(i.e. use the penultimate RAM frame for its stack), and its \verb+PC+ set to the
address of \verb+test+. Test is a supplied function/process that will help you
debug your nucleus. One can assign a variable (i.e. the \verb+PC+) the address of
a function by using
\begin{verbatim}
					. . . = (memaddr)test
\end{verbatim}
where \verb+memaddr+, in \verb+const.h+ has been aliased to \verb+unsigned int+.
Remember to declare the test function as “external” in your program by
including the line:
\begin{verbatim}
extern void test();
\end{verbatim}
\item Set-up the Idle Process and add it to its proper Ready Queue.
\item Call the scheduler.
\end{enumerate}
Once main() calls the scheduler its task is completed since control will never
return to main(). At this point the only mechanism for re-entering the nucleus
is through an exception; which includes device interrupts. As long as there are
processes to run, the processor is executing instructions on their behalf and only
temporarily enters the nucleus long enough to handle the device interrupts and
exceptions when they occur.
At boot/reset time the nucleus is loaded into RAM beginning with the second frame of RAM; 
0x0000.8000. The first frame of RAM is the ROM Reserved
Frame. Furthermore, the processor will be in
kernel-mode with virtual memory disabled and all interrupts masked. 
The \verb+PC+ is assigned 0x0000.8000 and the \verb+SP+, which
was initially set to RAMTOP at boot-time, will now be some value less than
RAMTOP due to the activation record for \verb+main()+ that now sits on the stack.
\subsection{SYS/Bp Exception Handling}
A \verb+SYSCALL+ or Breakpoint exception occurs when a \verb+SYSCALL+ or
\verb+BREAK+ assembler instruction is executed. Assuming that the SYS/Bp New Area in the ROM
Reserved Frame was correctly initialized during nucleus initialization, then after
the processor's and ROM-Excpt handler's actions when a \verb+SYSCALL+ or Breakpoint exception is raised, execution continues with the nucleus's SYS/Bp exception handler.
A \verb+SYSCALL+ exception is distinguished from a Breakpoint exception by the
value of the \verb+SWI+ instruction which raised the exception.
\verb+SYSCALL+ exceptions
are recognized via an exception code of Sys (8) while Breakpoint exceptions are
recognized via an exception code of Bp (9).
By convention the executing process places appropriate values in user registers
\verb+a1+-\verb+a4+ immediately prior to executing a \verb+SYSCALL+ or
\verb+BREAK+ instruction. The
nucleus will then perform some service on behalf of the process executing the
\verb+SYSCALL+ or \verb+BREAK+ instruction depending on the value found in \verb+a1+.
In particular, if the process making a \verb+SYSCALL+ request was in kernel-mode
and \verb+a1+ contained a value in the range \verb+[1..10]+ then the nucleus should 
perform one of the services described below.
\subsubsection{Create Process (SYS1)}
When requested, this service causes a new process, said to be a progeny of the
caller, to be created. \verb+a2+ should contain the physical address of a processor state
area at the time this instruction is executed and \verb+a3+ should contain
the priority level (\verb+PRIO_LOW+, \verb+PRIO_NORM+, \verb+PRIO_HIGH+). 
This processor state should be used
as the initial state for the newly created process. The process requesting the SYS1
service continues to exist and to execute. If the new process cannot be created due
to lack of resources (for example no more free ProcBlk's) or the priority level is not an acceptable value (including \verb+PRIO_IDLE+), an error code of
-1 is placed/returned in the caller's \verb+a1+, otherwise, return the PID
of the newly created process in \verb+a1+.
The SYS1 service is requested by the calling process by placing the value
1 in \verb+a1+, the physical address of a processor state in \verb+a2+, the priority level in \verb+a3+, and then executing a
\verb+SYSCALL+ instruction.
The following C code can be used to request a SYS1:
\begin{verbatim}
int SYSCALL (CREATEPROCESS, state t *statep, priority_enum prio)
\end{verbatim}
	Where the mnemonic constant CREATEPROCESS has the value of 1.
\subsubsection{Terminate Process (SYS2)}
This services causes the executing process or one process in its progeny to cease to exist. 
In addition, recursively, 
all progeny of the designated process are terminated as well. 
Execution of this instruction does not complete until all progeny are terminated.
The SYS2 service is requested by the calling process by placing the value 2 in
\verb+a1+, and the value of the designated process' PID in \verb+a2+ and
then executing a \verb+SYSCALL+ instruction.
When \verb+a2+ is set to zero, SYS2 terminates the calling process.
The following C code can be used to request a SYS2:
\begin{verbatim}
void SYSCALL (TERMINATEPROCESS, pid_t pid)
\end{verbatim}
Where the mnemonic constant TERMINATEPROCESS has the value of 2.
\subsubsection{Verhogen (V) (SYS3)}
When this service is requested, it is interpreted by the nucleus as a request to
perform a weighted V operation on a semaphore.
The V or SYS3 service is requested by the calling process by placing the value
3 in \verb+a1+, the physical address of the semaphore to be V'ed in \verb+a2+, 
the weight value (number of resources to be freed) in \verb+a3+, and then executing
a \verb+SYSCALL+ instruction.
The following C code can be used to request a SYS3:
\begin{verbatim}
void SYSCALL (VERHOGEN, int *semaddr, int weight)
\end{verbatim}
Where the mnemonic constant VERHOGEN has the value of 3.
\subsubsection{Passeren (P) (SYS4)}
When this services is requested, it is interpreted by the nucleus as a request to
perform a weighted P operation on a semaphore.
The P or SYS4 service is requested by the calling process by placing the value
4 in \verb+a1+, the physical address of the semaphore to be P'ed in \verb+a2+, 
the weight value (number of resources to be allocated) in \verb+a3+,
and then executing a \verb+SYSCALL+ instruction.
The following C code can be used to request a SYS4:
\begin{verbatim}
void SYSCALL (PASSEREN, int *semaddr, int weight)
\end{verbatim}
	Where the mnemonic constant PASSEREN has the value of 4.
\subsubsection{Specify Exception State Vector (SYS5)}
When this service is requested, the kernel stores the pointers to
six processor state areas for exception pass-up.
The SYS5 service is requested by the calling process by placing the value 5
in \verb+a1+, and an array of six processor state pointers in \verb+a2+.
These pointers are the addresses of the OLD area for TLB exceptions,
the NEW area for TLB exceptions, the OLD and NEW areas for SYSCALL and
the OLD and NEW area for Program Trap, respectively.
When an exception occurs for which an Exception State Vector has been specified
for, the nucleus stores the processor state at the time of the exception in the area
pointed to by the address in the specific OLD area, and loads the new processor state from the area
pointed to by the address given in the corresponding NEW area.
A process may request SYS5 service at most once.
An attempt to request a SYS5 service more than once by any process should be construed as an error and treated as a SYS2
of the process itself.
If an exception occurs while running a process which has not specified an
Exception State Vector for that exception type, then the nucleus should treat the
exception as a SYS2 as well.
The following C code can be used to request a SYS5:
\begin{verbatim}
void SYSCALL (SPECTRAPVEC, state t **state_vector)
\end{verbatim}
Where the mnemonic constant SPECTRAPVEC has the value of 5.
\subsubsection{Get CPU Time (SYS6)}
When this service is requested, it causes the processor time (in microseconds) used
by the requesting process, both as global time and user time, to be
returned to the calling process (global time is the time spent by
the processor for the calling process, including kernel time and user
time).
This means that the nucleus must record (in the ProcBlk) the amount of processor time used by
each process.
The SYS6 service is requested by the calling process by placing the value 6 in
\verb+a1+, the address of a \verb+cputime_t+ variable in \verb+a2+, the
address of a second \verb+cputime_t+ variable in \verb+a3+ and then
executing a \verb+SYSCALL+ instruction.
At SYS6 completion, the global processor time should be
stored at the address specified as \verb+a2+ while the processor time in
user time should be stored at the address specified as \verb+a3+.
The following C code can be used to request a SYS6:
\begin{verbatim}
void SYSCALL (GETCPUTIME, cputime_t *global, cputime_t *user)
\end{verbatim}
	Where the mnemonic constant GETCPUTIME has the value of 6.
\subsubsection{Wait For Clock (SYS7)}
This instruction performs a P operation on the nucleus maintained pseudo-clock
timer semaphore. This semaphore is V'ed every 100 milliseconds automatically
by the nucleus.
The SYS7 service is requested by the calling process by placing the value 7 in
\verb+a1+ and then executing a \verb+SYSCALL+ instruction.
The following C code can be used to request a SYS7:
\begin{verbatim}
void SYSCALL (WAITCLOCK)
\end{verbatim}
	Where the mnemonic constant WAITCLOCK has the value of 7.
\subsubsection{Wait for IO Device (SYS8)}
This service performs a P operation on the semaphore that the nucleus maintains
for the I/O device indicated by the values in \verb+a2+, \verb+a3+, and optionally
\verb+a4+. (P and V operation on I/O semaphores have always weight equal to
1).
Note that terminal devices are two independent sub-devices, and are handled by the SYS8 service as two independent devices. Hence
each terminal device has two nucleus maintained semaphores for it; one for character receipt and one for character transmission.
As discussed in Section 3.6 the nucleus will perform a V operation on the
nucleus maintained semaphore whenever that (sub)device generates an interrupt.
Once the process resumes after the occurrence of the anticipated interrupt,
the (sub)device's status word is returned in \verb+a1+. For character transmission and
receipt, the status word, in addition to containing a device completion code, will
also contain the character transmitted or received.
As described below in Section 3.6 it is possible that the interrupt can occur
prior to the request for the SYS8 service. In this case the requesting process will
not block as a result of the P operation and the interrupting device's status word,
which was stored off, can now be placed in \verb+a1+ prior to resuming execution.
The SYS8 service is requested by the calling process by placing the value 8
in \verb+a1+, the interrupt line number in \verb+a2+, the device number in
\verb+a3+ $([0. . .7])$ , TRUE
or FALSE in \verb+a4+ to indicate if waiting for a terminal read operation, and then
executing a \verb+SYSCALL+ instruction.
The following C code can be used to request a SYS8:
\begin{verbatim}
unsigned int SYSCALL (WAITIO, int intlNo, int dnum,
		int waitForTermRead)
\end{verbatim}
Where the mnemonic constant WAITIO has the value of 8.
\subsubsection{Get Process ID (SYS9)}
This service returns the Process ID of the caller by placing its value in
\verb+a1+.
The SYS9 service is requested by the calling process by placing the value
9, in \verb+a1+.
The following C code can be used to request a SYS9:
\begin{verbatim}
pid_t SYSCALL(GETPID)
\end{verbatim}
Where the mnemonic constant GETPID has the value of 9.
\subsubsection{Get Process Parent ID (SYS10)}
This service returns the Process ID of the caller's parent by placing its value in
\verb+a1+.
The SYS10 service is requested by the calling process by placing the value
10, in \verb+a1+.
The following C code can be used to request a SYS10:
\begin{verbatim}
pid_t SYSCALL(GETPPID)
\end{verbatim}
Where the mnemonic constant GETPPID has the value of 10.
\subsubsection{SYS1-SYS10 in User-Mode}
The above ten nucleus services are considered privileged services and are only
available to processes executing in kernel-mode. Any attempt to request one of
these services while in user-mode should trigger a PgmTrap exception response.
In particular JaeOS should simulate a PgmTrap exception when a privileged
service is requested in user-mode. This is done by moving the processor state
from the SYS/Bp Old Area to the PgmTrap Old Area, setting
\verb+CP15_Cause.ExcCode+ in
the PgmTrap Old Area to \verb+EXC_RESERVEDINSTR+ (Reserved Instruction), 
and calling JaeOS's PgmTrap exception handler.
\subsubsection{Breakpoint Exceptions and SYS11 and Above Exceptions}
The nucleus will directly handle all SYS1-SYS10 requests. The ROM-Excpt
handler will also directly handle some Breakpoint exceptions; those where
the requesting process was executing in kernel-mode and \verb+a1+ contained the code for
either \verb+LDST+, \verb+PANIC+, or \verb+HALT+.
For all other \verb+SYSCALL+ and Breakpoint exceptions the nucleus's SYS/Bp exception handler 
will take one of two actions depending on whether the offending
(i.e. Current) process has performed a SYS5:
\begin{itemize}
\item If the offending process has NOT issued a SYS5,
then the SYS/Bp exception should be handled like a SYS2: the current
process and all its progeny are terminated.
\item If the offending process has issued a SYS5, the handling of the SYS/Bp exception 
is “passed up.” The processor state is moved
from the SYS/Bp Old Area into the processor state area whose address was
recorded in the ProcBlk as the SYS/Bp Old Area Address. Finally, the processor 
state whose address was recorded in the ProcBlk as the SYS/Bp New
Area Address is made the current processor state.
\end{itemize}
\subsection{PgmTrap Exception Handling}
A PgmTrap exception occurs when the executing process attempts to perform
some illegal or undefined action. 
This includes all of the program trap types and reserved instructions. Assuming that the PgmTrap New Area in the ROM
Reserved Frame was correctly initialized during nucleus initialization, then after
the processor's and ROM-Excpt handler's actions when a PgmTrap exception is
raised, execution continues with the nucleus's PgmTrap exception handler. The
cause of the PgmTrap exception will be set in \verb+CP15_Cause.ExcCode+ in the PgmTrap
Old Area.
The nucleus's PgmTrap exception handler will take one of two actions depending on whether 
the offending (i.e. Current) process has performed a SYS5:
\begin{itemize}
\item If the offending process has NOT issued a SYS5,
then the PgmTrap exception should be handled like a SYS2: the current
process and all its progeny are terminated.
\item If the offending process has issued a SYS5, the
handling of the PgmTrap is “passed up.” The processor state is moved
from the PgmTrap Old Area into the processor state area whose address
was recorded in the ProcBlk as the PgmTrap Old Area Address. Finally, the
processor state whose address was recorded in the ProcBlk as the PgmTrap
New Area Address is made the current processor state.
\end{itemize}
\subsection{TLB Exception Handling}
A TLB exception occurs when $\mu$ARM fails in an attempt to translate a virtual
address into its corresponding physical address. 
Assuming that the TLB New Area in the ROM Reserved Frame was correctly initialized during nucleus initialization,
then after the processor's and ROM-Excpt handler's actions when a TLB exception is raised, execution continues with the nucleus's TLB exception handler. The
cause of the TLB exception will be set in \verb+CP15_Cause.ExcCode+ in the TLB Old Area.
The nucleus's TLB exception handler will take one of two actions depending
on whether the offending (i.e. Current) process has performed a SYS5:
\begin{itemize}
\item If the offending process has NOT issued a SYS5, then
the TLB exception should be handled like a SYS2: the current process and
all its progeny are terminated.
\item If the offending process has issued a SYS5, the handling
of the PgmTrap is “passed up.” The processor state is moved from the TLB
Old Area into the processor state area whose address was recorded in the
ProcBlk as the TLB Old Area Address. Finally, the processor state whose
address was recorded in the ProcBlk as the TLB New Area Address is made
the current processor state.
\end{itemize}
\subsection{Interrupt Exception Handling}
A device interrupt occurs when either a previously initiated I/O request completes
or when the Interval Timer makes a \verb+0x0000.0000+ $\rightarrow$ \verb+0xFFFF.FFFF+ transition. Assuming 
that the Ints New Area in the ROM Reserved Frame was 
correctly initialized during nucleus initialization, then after the
processor's and ROM-Excpt handler's actions when an Ints exception is raised, 
execution continues with the nucleus's Ints exception handler. Which interrupt lines
have pending interrupts is set in \verb+CP15_Cause.IP+. Furthermore,
for interrupt lines 3--7 the Interrupting Devices Bit Map, as defined in the
$\mu$ARM informal specifications document,
will indicate which devices on each of these interrupt lines have a pending interrupt. 
It is important to note that many devices per interrupt line may have an interrupt request pending, 
and that many interrupt lines may simultaneously be on.
Also, since each terminal device is two sub-devices, each terminal device may
have two pending interrupts simultaneously as well. You are strongly encouraged
to process only one interrupt at a time: the interrupt with the highest priority. The
	lower the interrupt line and device number, the higher the priority of the interrupt.
	When there are multiple interrupts pending, and the Interrupt exception handler
	only processes the single highest priority pending interrupt, 
the Interrupt exception handler will be immediately re-entered as soon as interrupts are unmasked
	again; effectively forming a loop until all the pending interrupts are processed.
	As described in Section 5.7-pops, terminal devices are actually two sub-devices;
	a transmitter and a receiver. These two sub-devices operate independently and
	concurrently. Both sub-devices may have an interrupt pending simultaneously.
	For purposes of prioritizing pending interrupts, terminal transmission (i.e. writing
			to the terminal) is of higher priority than terminal receipt (i.e. reading from the
				terminal). Hence the processor Interval Timer (interrupt line 2) is the highest priority interrupt and reading from terminal 7 (interrupt line 7, device 7; read) is the
			lowest priority interrupt.
			The nucleus's Interrupts exception handler will perform a number of tasks:
\begin{itemize}
	\item Acknowledge the outstanding interrupt. For all devices except the Interval Timer
	this is accomplished by writing the acknowledge command code in the interrupting device's \verb+COMMAND+ device register. Alternatively, writing a new command in
	the interrupting device's device register will also acknowledge the interrupt.
	An interrupt for a timer device is acknowledged by loading the timer with a
	new value.
	\item Perform a V operation (weight 1) on the nucleus maintained semaphore associated
	with the interrupting (sub)device if the semaphore has value less than 1. The nucleus maintains two semaphores
	for each terminal sub-device. For Interval Timer interrupts that represent a
	pseudo-clock tick (see Section 3.7.1), perform the V operation on the nucleus maintained pseudo-clock timer semaphore.
	A V operation on the pseudo clock should unlock all the waiting processes.
	\item If the SYS8 for this interrupt was requested prior to the handling of this
	interrupt, recognized by the V operation above unblocking a blocked process, store the interrupting (sub)device's status word in the newly unblocked
	process's \verb+a1+. If the SYS8 for this interrupt has not yet been requested,
	recognized by the V operation not unblocking any process, store off the
	interrupting device's status word until the SYS8 is eventually requested.
\end{itemize}
\subsection{Nuts and Bolts}
\subsubsection{Timing Issues}
	While $\mu$ARM has two clocks, the TOD clock and Interval Timer, only the
Interval Timer can generate interrupts. Hence the Interval Timer must be
used simultaneously for two purposes: generating interrupts to signal the
end
of processes' time slices, and to signal the end of each 100 millisecond
period
(a pseudo-clock tick); i.e. the time to V the semaphore associated with the
pseudo-clock timer.
It is insufficient to simply V the pseudo-clock timer's semaphore after
every 20 time slices; processes may block (SYS4, SYS7, or SYS8) or terminate (SYS2) long before the end of their current time slice. A more careful
accounting method is called for; one where some (most) of the Interval
Timer interrupts represent the conclusion of a time slice while others
represent the conclusion of a pseudo-clock tick.
Furthermore, when no process requests a SYS7, the pseudo-clock timer
semaphore, if left unadjusted, will grow by 1 every 100 milliseconds. This
means that if a process, after 500 milliseconds requests a SYS7, and there
were no intervening SYS7 requests, it will not block until the next pseudo-clock tick as hoped for, but will immediately resume its execution stream.
Therefore at each pseudo-clock tick, if no process was unblocked by the V
operation (i.e. the semaphore's value after the increment performed during
the V operation was greater than zero), the semaphore's value should be
decremented by one (i.e. reset to zero).
The opposite is also true; if more than one process requests a SYS7 in
between two adjacent pseudo-clock ticks then at the next pseudo-clock tick,
all of the waiting processes should be unblocked and not just the process
that was waiting the longest.
The CPU time used by each process must also be kept track of (i.e. SYS6).
This implies an additional field to be added to the ProcBlk structure.
While the Interval Timer is useful for generating interrupts, the TOD clock
is useful for recording the length of an interval. By storing off the TOD
clock's value at both the start and end of an interval, one can compute the
duration of that interval.
The Interval Timer and TOD clock are mechanisms for implementing
JaeOS's policy. Timing policy questions that need to be worked out include:
\begin{itemize}
\item While the time spent by the nucleus handling an I/O or Interval Timer
interrupt needs to be measured for pseudo-clock tick purposes, which
process, if any, should be “charged” with this time. Note: it is possible
for an I/O or Interval Timer interrupt to occur even when there is no
current process.
\item While the time spent by the nucleus handling a \verb+SYSCALL+ request
needs to be measured for pseudo-clock tick purposes, which process, if
any, should be “charged” with this time.
\end{itemize}
\subsubsection{Returning from a SYS/Bp Exception}
\verb+SYSCALL+'s that do not result in process termination return control to the requesting process's execution stream. 
This is done either immediately (e.g. SYS6) or
after the process is blocked and eventually unblocked (e.g. SYS8). In any event
the \verb+PC+ that was saved is, for this kind of exceptions, the address of the address of the \verb+SYSCALL+ assembly instruction
that caused the exception.
\subsubsection{Loading a New Processor State}
It is the job of the ROM-Excpt handler to load new processor states; either as
part of “passing up” exception handling (the loading of the processor state from
the appropriate New Area) or for \verb+LDST+ processing. 
Whenever the ROM-Excpt handler loads a processor state, the previous state
of the running process get stored in the corresponding OLD area.
Any information concerning the running process (prior to the exception) should be retrieved from
that OLD state, e.g. the CPRS register fields are useful to state the
running execution mode.
\subsubsection{Process Termination}
When a process is terminated there is actually a whole (sub)tree of processes that
get terminated. There are a number of tasks that must be accomplished:
\begin{itemize}
	\item The root of the sub-tree of terminated processes must be “orphaned” from
	its parents; its parent can no longer have this ProcBlk as one of its progeny.
	\item If the value of a semaphore is negative, it means that some processes (of which the ProcBlks are blocked on that semaphore) have requested a number of resources. Hence if a terminated process is blocked on a semaphore, the
	value of the semaphore must be adjusted.
	\item If a terminated process is blocked on a device semaphore, the semaphore
	should NOT be adjusted. When the interrupt eventually occurs the semaphore
	will get V'ed by the interrupt handler.
	\item The process count and soft-blocked variables need to be adjusted accordingly.
	\item Processes (i.e. ProcBlk's) can't hide. A ProcBlk is either the current process, sitting on a ready queue, blocked on a device semaphore, or blocked on a non-device semaphore.
\end{itemize}
\subsubsection{Module Decomposition}
	One possible module decomposition is as follows:
\begin{enumerate}
	\item{\verb+initial.c+}
	This module implements main() and exports the nucleus's global
variables. (e.g. Process Count, device semaphores, etc.)
	\item{\verb+interrupts.c+} This module implements the device interrupt exception
	handler. This module will process all the device interrupts, including Interval Timer interrupts, converting device interrupts into V operations on
	the appropriate semaphores.
\item{\verb+exceptions.c+}
	This module implements the TLB, PgmTrap and SYS/Bp
	exception handlers.
\item{\verb+scheduler.c+}
	This module implements JaeOS's process scheduler and deadlock detector.
\end{enumerate}
